\documentclass[12pt,a4paper]{article}
\usepackage[utf8]{inputenc}
\usepackage[russian]{babel}
\usepackage[OT1]{fontenc}
\usepackage{amsmath}
\usepackage{amsfonts}
\usepackage{amssymb}
\usepackage{graphicx}
\graphicspath{{Images/}}
\usepackage[left=2cm,right=2cm,top=2cm,bottom=2cm]{geometry}
\usepackage{calc}
\usepackage{wrapfig}
\usepackage{setspace}
\usepackage{indentfirst}
\usepackage{subfigure}
\usepackage{wasysym}

\title{
Отчет по скрещиванию дроздофил
}

\author{Мыздриков Иван Витальевич}
\date{\today}
\begin{document}

\maketitle
\newpage


\section{Методика скрещивания}
Первая линия скрещивания началась с 4-ех мух - 2 самца дикого типа с серым телом и красными и 2 самки "нахабиных" с белыми глазами и желтым теломб все мухи были из чистых линий (без гетерозигот).
Вторая линия была посажена на скрещивания Роминой, она скрещивала (в том числе) 2 самца с белыми глазами и кривыми крыльями с 2-мя самками с нормальными крыльями и белыми глазами, также чистая линия.
После получения F1 обе линии скрещивались для получения F2. Стоить заметить, что в скрещиваниях не было контаминации между поколениями, так как до выхода следующего поколения предыдущее убиралось из пробирки.





\section{Результаты F1}
В обоих случаях выполнялся 1 закон Менделя - единообразие потомков первого поколения.
Закон формулируется как "при скрещивании двух гомозиготеых организмов относящихся к чистым линиям и отличающихся друг от друга по одной паре альтернативных проявлений все поколение F1 будет нести признаки одного из родителелей.
В случае со скрещиванием с нахабиным (обозначим как 1-ая линия) закон выполняется немного по другому принипую Оба, гена ассоцирированные с мутациями, связаны и располагаются на половой хромосоме Х.
Тогда самки этого скрещивания имеют такой набор: $X^{YW} X^{YW}$, а самцы - $Y X^{NN}$, где $N$ - нормальный признак, а и $Y W$- соответственно аллели желтого тела и белых глаз.
При таких генотипах могут быть только такие комбинации гамет: $X^{YW} Y$ и $X^{YW} X^{NN}$. Что согласуется с нашими данными, а два самца c белыми глазами и нормальным телом проблематично обьяснить, так как кроссинговер не влияет в данном случае на гаметы.
Тогда эти самцы либо пробрались в пробирку для скрещивания или лежали в морилке (то есть не имеют отношения к скрещиванию), либо были неправильно определены фенотипы и это самцы с желтым телом.
       
\begin{center}
    \begin{tabular}{|c|c|c|c|}  \hline
        \multicolumn{4}{|c|}{Фенотипы F1 1-ой линии} \\ \hline
        Пол & Тело   & Глаза   & Кол-во \\ \hline
        M   & Желтое & Белые   & 43 \\ \hline
        F   & Серое  & Красные & 55 \\ \hline
        M   & Серое  & Белые & 2 \\ \hline

    \end{tabular}
\end{center}

Поскольку есть разница в количестве особей разного пола, стоит проверить по $Х^2$ расхождение и объяснить принцып работы этого метода.
Признаки потомков хоть и случайны, но подчиняются опреденным закономерностям и можно определить вероятность каждого признака. В случае с полом это опрделяется набором половых хромосом слившихся гамет.
Метод "Хи-квадрат" позволяет определить с какой вероятностью наши результаты (эмпирические данные) укладываются в теорию, описывается формулой:
\begin{equation}
    X^2 = \sum \frac{(E-Q)^2}{E}
\end{equation}
Где суммируются случаи для каждого фенотипа по интересующему признаку.
Приведем расчеты для полов в F1 без учета 2-ух самцов или с учетом их как самцов с фенотипом $WY$
\newline Без учета:
\begin{equation}
    X^2 = \sum \frac{(E-Q)^2}{E} = \frac{(43-49)^2}{49} +\frac{(55-49)^2}{49} \approx 1.47
\end{equation}
\newline С учетом как фенотип $WY$:
\begin{equation}
    X^2 = \sum \frac{(E-Q)^2}{E} = \frac{(45-50)^2}{50} +\frac{(55-50)^2}{50} = 1
\end{equation}
В обоих случаях будет степень свободы 1, а для нее и p=0.01 критическое значение 6.635, в обоих случаях закон подтвердился.
Провести расчеты с учетом лишних самцов с их настоящим фенотипом невозможно, так как будет деление на 0.






\section{Результаты F2}
Для первой линии F2 получить не удалось, либо из-за отравления мух никотином, либо из-за неизвестным мутаций, приведших к бесплодности. Потому будем рассматривать только 2-ую линию.
\begin{center}
    \begin{tabular}{|c|c|c|c|}  \hline
        \multicolumn{4}{|c|}{Фенотипы F2 2-ой линии} \\ \hline
        Пол & Крылья  & Глаза   & Кол-во \\ \hline
        M   & Кривые  & Белые   & 12 \\ \hline
        F   & Прямые  & Белые   & 35 \\ \hline
        M   & Прямые  & Белые   & 46 \\ \hline
        F   & Кривые  & Белые   & 19 \\ \hline
    \end{tabular}
\end{center}
Проверим по $X^2$ правдивость рапспределения полов и феноипов.
Всего 112 особей, из них 60 должно приходится на самцов, $E = 56$. Так как пизнак кривых крыльев не связан с полом и в 
первом скрещевыание было 2 чистые линии, то по 2 закону менделя должно быть расщепление 3:1 доминантного к рецессивному признаку. Тогда должно быть 84 особей 
с кривыми крыльями и 28 с прямыми (нормальными).
Проведем расчеты $X^2$ для полов:
\begin{equation}
    X^2 = \sum \frac{(E-Q)^2}{E} = \frac{(46+12-50)^2 + (35 + 19 - 50)^2}{56} \approx 1.43
\end{equation}
И для признаков:
\begin{equation}
    X^2 = \sum \frac{(E-Q)^2}{E} = \frac{(46+35-84)^2}{84} +\frac{(19+12 -28)^2}{28} \approx 0.43
\end{equation}
Оба этих значения меньшше, чем соответствующие значения для уровня значимости p=0.01 при одной степени свободы, закон выполняется.

\section{Выводы}
В ходе проведенных экспериментов мы эмпирически подтвердили 1 и 2 законы менделя, нашли потенциально летальную комбинацию мутаций.
Одна из тем, незатронутых в отчете - кроссинговер. Если бы удалось вывести F2 2-ой линии и, то можно было изучить этот процесс.
Проведя мысленный эксперимент, взяв аллели G и R как доминантные, а g и r рециссивными (гены отвечают за лкруаску тела и глаз), тогда у родителей следующие генотипы: 
\female $X^{gr}X^{gr}$ и \mars $YX^{GR}$
\newline Тогда потомки F1 будут иметь \female $X^{GR}X^{gr}$ и \mars $YX^{gr}$, если взять, что гены сцеплены, то F2 уже будет иметь генотипы 
\female $X^{GR}X^{gr}$, \female $X^{gr}X^{gr}$, \mars $YX^{GR}$ и \mars $YX^{gr}$, но кроссинговер бы привет к гаметам $X^{Gr}$ и $X^{gR}$.
Тогда появились бы мухи с генотипом $X^{Gr}X^{gr}$ и $X^{gR}X^{gr}$ или же скамкам с красными глазамиб желтым телом и самки с белыми глазами, серым телом.


\end{document}